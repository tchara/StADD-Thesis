%
% Duale Hochschule Sachsen
% — Staatliche Studienakademie Dresden
%
% LaTeX-Vorlage für Abschlussarbeiten im Bereich Technik
%
%
% Erstellt von Tenshi Hara und bereitgestellt unter einer
%
%       GNU Public License, Version 3 (GPLv3)
%
% Näheres zur Lizenz entnehmen Sie der Datei Lizenz.txt.
% Sollten Sie diese Vorlage ohne die Lizenz-Datei erhalten
% haben, informieren Sie bitte die Free Software Foundation.
%
%%%%%%%%%%%%%%%%%%%%%%%%%%%%%%%%%%%%%%%%%%%%%%%%%%%%%%%%%%%%%%
%
% Vorlage nutzt BibTeX. Falls Sie BibLaTeX benutzen wollen,
% scrollen Sie ans Ende dieser Datei und beachten Sie die
% Hinweise dort.
%
%%%%%%%%%%%%%%%%%%%%%%%%%%%%%%%%%%%%%%%%%%%%%%%%%%%%%%%%%%%%%%
%
% Zufrieden? ---> https://ko-fi.com/tchara
%
%%%%%%%%%%%%%%%%%%%%%%%%%%%%%%%%%%%%%%%%%%%%%%%%%%%%%%%%%%%%%%
%
% 1. Personendaten
%
\newcommand{\myTitle}{Empfehlung für die Gestaltung wissenschaftlicher Ausarbeitungen im Studienbereich Technik, insbesondere den Ingenieurwissenschaften}
\newcommand{\mySubtitle}{Vorlage für die Bachelor-Arbeit}
\newcommand{\myDegree}{Bachelor of Engineering}
\newcommand{\myDegreeShort}{B.Eng.}
\newcommand{\myName}{Paris Studiumsmensch}
\newcommand{\myCompany}{Praxisgesellschaft mbH \& Co KG}
\newcommand{\myCompanyAddress}{Musterstraße 1, 01001 Dresden}
\newcommand{\myId}{3012345}
\newcommand{\myExpert}{Dipl.-Ing. Kader Praxismensch}
\newcommand{\myProf}{Prof. Dr.-Ing. Arin Schlaumeier}
\newcommand{\myAcademy}{Staatliche Studienakademie Dresden}
% Vorsicht mit den Begriffen „Studiengang“ und „Studienrichtung“!
\newcommand{\myStudy}{Studienrgang Informationstechnik}
\newcommand{\myLocation}{Dresden} 
\newcommand{\myVersion}{Version 1.0}
% Abgabedatum
\newcommand{\mySubmissionDate}{22. Juli 202X}
% Datum der Selbständigkeitserklärung
% Abgabedatum und Datum der Selbständigkeitserklärung müssen nicht
% zwingend identisch sein. Sie können die Erklärung bereits ein paar
% Tage vorher (beispielsweise direkt nach dem Drucken und Binden)
% unterschreiben, während Sie die Arbeit erst später einreichen.
\newcommand{\myDeclarationDate}{20. Juli 202X}

%
% 2. Konfiguration
%
% 2a) Hat diese Arbeit einen Sperrvermerk?
%
\newboolean{lockflag}
\setboolean{lockflag}{true} % „false“, falls kein Sperrvermerk
%
% 2b) Soll statt des Abstracts ein Autorenreferat erscheinen?
%
\newboolean{extendedabstract}
\setboolean{extendedabstract}{false} % „true“, falls Autorenreferat
%
% 2c) Soll die Selbständigkeitserklärung vor das Inhaltsverzeichnis?
%     (Manche Gutachter wollen, dass die Selbständigkeitserklärung
%      am Anfang des Dokuments (direkt nach dem Titelblatt) steht.)
%
\newboolean{frontdeclaration}
\setboolean{frontdeclaration}{false} % „true“, für vorne

%
% 3. gescanntes Auftragsblatt einfügen
%
\newboolean{auftragsblatt}
\setboolean{auftragsblatt}{true}
% "false", falls kein Auftragsblatt eingebaut werden soll.
% Das Auftragsblatt muss sowohl im Printexemplar als auch in der
% elektronischen Version eingebunden werden! - In dieser Vorlage
% ist dafür der Platz zwischen der Zusammenfassung und dem
% Inhaltsverzeichnis vorgesehen.
% Für die elektronische Version bietet sich das Einbinden eines
% Scans an (true).
% Für die Printexemplare sollten Sie die Seitenzähler einfach nur
% um 2 erhöhen (true) und dann später die entsprechenden Blätter
% einlegen:
% * das Pflichtexemplar für das Archiv erhält das Original
% * die anderen Exemplare eine Kopie

%
% 4. Lorem Ipsum (am Ende auskommentieren!)
%
\usepackage{lipsum}

%
% 5. zu verwendende Pakete
%
% Hinweise zur Geometrie im Dokument beachten!
\usepackage{geometry}
\geometry{
	inner=40mm,
	outer=30mm,
	top=30mm
}
% Setzen Sie im folgenden die Option „htt“, wenn Sie wollen, dass
% Text in den Umgebungen \texttt und \ttfamily auch den Trennregeln
% des hyphenat-Pakets unterliegen sollen.
%\usepackage[htt]{hyphenat}
\usepackage{hyphenat}
\usepackage[ngerman,english]{babel}
\usepackage[babel]{csquotes}
\usepackage{cite}
\usepackage{amsmath,amssymb,amstext,amsthm}
\usepackage{upgreek,units}
\usepackage{listings}
\lstset{ %
	basicstyle=\small,		% Schriftgröße im Kodeblock
	numberstyle=\tiny,
	numbers=left,			% Position der Zeilennummern
	stepnumber=1,			% Schrittweite der Zeilennummern („1“ bedeutet jede Zeile wird beschriftet)
	frame=tbl,				% Rahmen um Kodeblock (Standard: „single“)
	xleftmargin=15pt,
	numbersep=10pt,
	tabsize=4,				% Tabulatorgröße (4 Leerzeichen)
	captionpos=t,			% Position der Beschriftung
	breaklines=true,		% automatischer Zeilenumbruch
	breakatwhitespace=true,	% automatischer Zeilenumbruch nur an Leerzeichen, Tabulatoren, etc.
}
\renewcommand{\lstlistingname}{Quellkode}
\lstdefinestyle{plain}{frame=single, breaklines=true}
\lstdefinestyle{Java}{language=Java, basicstyle=\small, numbers=left, numberstyle=\tiny, numberblanklines=false, tabsize=4, xleftmargin=1.6em, frame=single, breaklines=true}
\usepackage{graphicx}
\usepackage{caption,subcaption}

\usepackage[hang,multiple]{footmisc}
\renewcommand{\footnotemargin}{1.2em}
\usepackage{longtable,booktabs,multirow,colortbl}
\usepackage{rotating}
\usepackage{remreset}
\usepackage{savefnmark}
\usepackage{enumerate}
\usepackage[shortlabels]{enumitem}
\setlist{noitemsep, itemsep=3pt, topsep=0pt, partopsep=6pt}
\setenumerate{noitemsep, itemsep=3pt, topsep=0pt, partopsep=6pt}

\usepackage{thmtools}
\usepackage[nothm]{thmbox}
\usepackage[mathscr]{eucal}
\newtheoremstyle{style}
	{0}									%	Platz drüber
	{0}									%	Platz drunter
	{\normalfont}						%	Schriftstil
	{}									%	Einrückung
	{\normalfont\bfseries}				%	Theorembezeichner-Stil
	{:\\\vphantom{-}\vspace*{-.75em}\\}	%	Trennzeichen Bezeichner --> Inhalt
	{5pt plus 1pt minus 1pt}			%	Trennzeichen Text --> Theorem
	{{\normalfont\bfseries \thmname{#1}\thmnumber{ #2}\thmnote{ (#3)}}}
\theoremstyle{style}

\definecolor{HKS41-100}{cmyk}{1.0, 0.7, 0.1, 0.5}
\definecolor{HKS41-90}{cmyk}{0.9, 0.63, 0.09, 0.45}
\definecolor{HKS41-80}{cmyk}{0.8, 0.56, 0.08, 0.4}
\definecolor{HKS41-70}{cmyk}{0.7, 0.49, 0.07, 0.35}
\definecolor{HKS41-60}{cmyk}{0.6, 0.42, 0.06, 0.3}
\definecolor{HKS41-50}{cmyk}{0.5, 0.35, 0.05, 0.25}
\definecolor{HKS41-40}{cmyk}{0.4, 0.28, 0.04, 0.2}
\definecolor{HKS41-30}{cmyk}{0.3, 0.21, 0.03, 0.15}
\definecolor{HKS41-20}{cmyk}{0.2, 0.14, 0.02, 0.1}
\definecolor{HKS41-10}{cmyk}{0.1, 0.07, 0.01, 0.05}
\definecolor{HKS44-100}{cmyk}{1.0, 0.5, 0.0, 0.0}
\definecolor{HKS44-90}{cmyk}{0.9, 0.45, 0.0, 0.0}
\definecolor{HKS44-80}{cmyk}{0.8, 0.4, 0.0, 0.0}
\definecolor{HKS44-70}{cmyk}{0.7, 0.35, 0.0, 0.0}
\definecolor{HKS44-60}{cmyk}{0.6, 0.3, 0.0, 0.0}
\definecolor{HKS44-50}{cmyk}{0.5, 0.25, 0.0, 0.0}
\definecolor{HKS44-40}{cmyk}{0.4, 0.2, 0.0, 0.0}
\definecolor{HKS44-30}{cmyk}{0.3, 0.15, 0.0, 0.0}
\definecolor{HKS44-20}{cmyk}{0.2, 0.1, 0.0, 0.0}
\definecolor{HKS44-10}{cmyk}{0.1, 0.05, 0.0, 0.0}
\definecolor{HKS36-100}{cmyk}{0.8, 0.9, 0.0, 0.0}
\definecolor{HKS36-90}{cmyk}{0.72, 0.81, 0.0, 0.0}
\definecolor{HKS36-80}{cmyk}{0.64, 0.72, 0.0, 0.0}
\definecolor{HKS36-70}{cmyk}{0.56, 0.63, 0.0, 0.0}
\definecolor{HKS36-60}{cmyk}{0.48, 0.54, 0.0, 0.0}
\definecolor{HKS36-50}{cmyk}{0.4, 0.45, 0.0, 0.0}
\definecolor{HKS36-40}{cmyk}{0.32, 0.36, 0.0, 0.0}
\definecolor{HKS36-30}{cmyk}{0.24, 0.27, 0.0, 0.0}
\definecolor{HKS36-20}{cmyk}{0.16, 0.18, 0.0, 0.0}
\definecolor{HKS36-10}{cmyk}{0.08, 0.09, 0.0, 0.0}
\definecolor{HKS33-100}{cmyk}{0.5, 1.0, 0.0, 0.0}
\definecolor{HKS33-90}{cmyk}{0.45, 0.9, 0.0, 0.0}
\definecolor{HKS33-80}{cmyk}{0.4, 0.8, 0.0, 0.0}
\definecolor{HKS33-70}{cmyk}{0.35, 0.7, 0.0, 0.0}
\definecolor{HKS33-60}{cmyk}{0.3, 0.6, 0.0, 0.0}
\definecolor{HKS33-50}{cmyk}{0.25, 0.5, 0.0, 0.0}
\definecolor{HKS33-40}{cmyk}{0.2, 0.4, 0.0, 0.0}
\definecolor{HKS33-30}{cmyk}{0.15, 0.3, 0.0, 0.0}
\definecolor{HKS33-20}{cmyk}{0.1, 0.2, 0.0, 0.0}
\definecolor{HKS33-10}{cmyk}{0.05, 0.1, 0.0, 0.0}
\definecolor{HKS57-100}{cmyk}{1.0, 0.0, 0.9, 0.2}
\definecolor{HKS57-90}{cmyk}{0.9, 0.0, 0.81, 0.18}
\definecolor{HKS57-80}{cmyk}{0.8, 0.0, 0.72, 0.16}
\definecolor{HKS57-70}{cmyk}{0.7, 0.0, 0.63, 0.14}
\definecolor{HKS57-60}{cmyk}{0.6, 0.0, 0.54, 0.12}
\definecolor{HKS57-50}{cmyk}{0.5, 0.0, 0.45, 0.1}
\definecolor{HKS57-40}{cmyk}{0.4, 0.0, 0.36, 0.08}
\definecolor{HKS57-30}{cmyk}{0.3, 0.0, 0.27, 0.06}
\definecolor{HKS57-20}{cmyk}{0.2, 0.0, 0.18, 0.04}
\definecolor{HKS57-10}{cmyk}{0.1, 0.0, 0.09, 0.02}
\definecolor{HKS65-100}{cmyk}{0.65, 0.0, 1.0, 0.0}
\definecolor{HKS65-90}{cmyk}{0.585, 0.0, 0.9, 0.0}
\definecolor{HKS65-80}{cmyk}{0.52, 0.0, 0.8, 0.0}
\definecolor{HKS65-70}{cmyk}{0.455, 0.0, 0.7, 0.0}
\definecolor{HKS65-60}{cmyk}{0.39, 0.0, 0.6, 0.0}
\definecolor{HKS65-50}{cmyk}{0.325, 0.0, 0.5, 0.0}
\definecolor{HKS65-40}{cmyk}{0.26, 0.0, 0.4, 0.0}
\definecolor{HKS65-30}{cmyk}{0.195, 0.0, 0.3, 0.0}
\definecolor{HKS65-20}{cmyk}{0.13, 0.0, 0.2, 0.0}
\definecolor{HKS65-10}{cmyk}{0.065, 0.0, 0.1, 0.0}
\definecolor{HKS07-100}{cmyk}{0.0, 0.6, 1.0, 0.0}
\definecolor{HKS07-90}{cmyk}{0.0, 0.54, 0.9, 0.0}
\definecolor{HKS07-80}{cmyk}{0.0, 0.48, 0.8, 0.0}
\definecolor{HKS07-70}{cmyk}{0.0, 0.42, 0.7, 0.0}
\definecolor{HKS07-60}{cmyk}{0.0, 0.36, 0.6, 0.0}
\definecolor{HKS07-50}{cmyk}{0.0, 0.3, 0.5, 0.0}
\definecolor{HKS07-40}{cmyk}{0.0, 0.24, 0.4, 0.0}
\definecolor{HKS07-30}{cmyk}{0.0, 0.18, 0.3, 0.0}
\definecolor{HKS07-20}{cmyk}{0.0, 0.12, 0.2, 0.0}
\definecolor{HKS07-10}{cmyk}{0.0, 0.06, 0.1, 0.0}
\definecolor{HKS14-100}{cmyk}{0.0, 1.0, 1.0, 0.0}
% Benannte Farben
\definecolor{white}{gray}{1.00}
\definecolor{black}{gray}{0.00}
\definecolor{skyblue}{cmyk}{0.4, 0.2, 0.0, 0.0}             % HKS44-40
\definecolor{blue}{cmyk}{1.0, 0.5, 0.0, 0.0}                % HKS44-100
\definecolor{lightblue}{cmyk}{0.7, 0.35, 0.0, 0.0}          % HKS44-70
\definecolor{darkblue}{rgb}{0.04, 0.16, 0.32}               % 
\definecolor{extradarkblue}{cmyk}{1.00, 0.70, 0.10, 0.50}   % HKS41-100
\definecolor{darkgreen}{cmyk}{1.0, 0.0, 0.9, 0.2}           % HKS57-100
\definecolor{green}{cmyk}{0.65, 0.0, 1.0, 0.0}              % HKS65-100
\definecolor{purple}{cmyk}{0.5, 1.0, 0.0, 0.0}              % HKS33-100
\definecolor{indigo}{cmyk}{0.8, 0.9, 0.0, 0.0}              % HKS36-100
\definecolor{gray}{gray}{0.59}
\definecolor{darkgray}{gray}{0.50}
\definecolor{darkcyan}{cmyk}{0.87, 0.4, 0.4, 0.0}
\definecolor{cyan}{cmyk}{0.78, 0.19, 0.01, 0.0}
\definecolor{lightcyan}{cmyk}{0.39, 0.095, 0.005, 0.0}
\definecolor{extralightcyan}{cmyk}{0.16, 0.1, 0.0, 0.0}

\definecolor{shadeThe}{cmyk}{.1,.05,0,0}		% HKS44_10%
\definecolor{frameThe}{cmyk}{.7,.35,0,0}		% HKS44_70%
\definecolor{shadeDef}{cmyk}{.07,0,.1,0}		% HKS65_10%
\definecolor{frameDef}{cmyk}{.2,0,.4,0}			% HKS65_40%
\definecolor{shadeThm}{cmyk}{.05,.1,0,0}		% HKS33_10%
\definecolor{frameThm}{cmyk}{.2,.4,0,0}			% HKS33_40%
\definecolor{shadeLem}{cmyk}{0,.06,.1,0}		% HKS07_10%
\definecolor{frameLem}{cmyk}{0,.24,.4,0}		% HKS07_40%
\definecolor{shadeCor}{cmyk}{.1,.07,.01,.05}	% HKS41_10%
\definecolor{frameCor}{cmyk}{.4,.28,.04,.2}		% HKS41_40%
\definecolor{shadePrf}{cmyk}{.08,.09,0,0}		% HKS36_10%
\definecolor{framePrf}{cmyk}{.32,.36,0,0}		% HKS36_40%
\definecolor{shadeAss}{cmyk}{.1,0,.09,.02}		% HKS57_10%
\definecolor{frameAss}{cmyk}{.34,0,.36,.08}		% HKS57_40%

\declaretheorem[name=Arbeitsthese,shaded={rulecolor=frameThe,rulewidth=1pt,bgcolor=shadeThe,textwidth=\textwidth-2pt}]{myworkingthesis}
\declaretheorem[name=Annahme,numbered=no,shaded={rulecolor=frameAss,rulewidth=1pt,bgcolor=shadeAss,textwidth=\textwidth-2pt}]{myassertion}
\declaretheorem[numberwithin=section,name=Definition,shaded={rulecolor=frameDef,rulewidth=1pt,bgcolor=shadeDef,textwidth=\textwidth-2pt}]{mydef}
\declaretheorem[sibling=mydef,name=Theorem,shaded={rulecolor=frameThm,rulewidth=1pt,bgcolor=shadeThm,textwidth=\textwidth-2pt}]{mytheorem}
\declaretheorem[sibling=mydef,name=Lemma,shaded={rulecolor=frameLem,rulewidth=1pt,bgcolor=shadeLem,textwidth=\textwidth-2pt}]{mylemma}
\declaretheorem[sibling=mydef,name=Korollar,shaded={rulecolor=frameCor,rulewidth=1pt,bgcolor=shadeCor,textwidth=\textwidth-2pt}]{mycorollary}
\declaretheorem[sibling=mydef,name=Beweis,shaded={rulecolor=framePrf,rulewidth=1pt,bgcolor=shadePrf,textwidth=\textwidth-2pt}]{myproof}
\declaretheorem[sibling=mydef,name=Abbildung]{myfig}

\AtBeginDocument{\numberwithin{lstlisting}{section}}
\newcounter{thmhelper}[section]
\newcounter{thesiscount}
\makeatletter
	\@removefromreset{footnote}{chapter}
\makeatother

\usepackage{wrapfig}

\newcommand{\inlinecode}[1]{%
	\colorbox{extralightcyan}{%
		\texttt{#1}%
	}%
}%

\newcommand{\mytable}[0]{%
	\setcounter{table}{\thethmhelper}%
	\addtocounter{thmhelper}{1}%
}{%
}%
\newcommand{\myequation}[0]{%
	\setcounter{equation}{\thethmhelper}%
	\addtocounter{thmhelper}{1}%
}{%
}%
\newenvironment{workingthesis}[1][]{%
	\addtocounter{thesiscount}{1}%
	\begin{myworkingthesis}[#1]}{\end{myworkingthesis}%
}%
\newenvironment{assertion}[1][]{%
	\begin{myassertion}[#1]}{\end{myassertion}%
}%
\newenvironment{definition}[1][]{%
	\addtocounter{thmhelper}{1}%
	\begin{mydef}[#1]}{\end{mydef}%
}%
\newenvironment{theorem}[1][]{%
	\addtocounter{thmhelper}{1}%
	\begin{mytheorem}[#1]}{\end{mytheorem}%
}%
\newenvironment{lemma}[1][]{%
	\addtocounter{thmhelper}{1}%
	\begin{mylemma}[#1]}{\end{mylemma}%
}%
\newenvironment{corollary}[1][]{%
	\addtocounter{thmhelper}{1}%
	\begin{mycorollary}[#1]}{\end{mycorollary}%
}%
\renewenvironment{proof}[1][]{%
	\addtocounter{thmhelper}{1}%
	\begin{myproof}[#1]}{%
		\nopagebreak%
		\\%
		{} \qed%
	\end{myproof}%
}%
\newcommand{\comment}[1]{%
	\, {} \hfill {\, }%
	{\color{gray}\begin{thmbox}[S]{\textit{Kommentar}}%
		\setlength{\parindent}{-0.3em}%
		\textit{#1}%
	\end{thmbox}}%
	\vspace{1em}%
}%
\renewcommand{\example}[1]{%
	\, {} \hfill {\, }%
	\begin{thmbox}[S]{\textit{Beispiel}}%
		\setlength{\parindent}{-0.2em}%
		#1%
	\end{thmbox}%
	\vspace{1em}%
}%

% Paket catoptions ist defekt; für Optionen in ASCII-Zeichen
% muss händisch korrigiert werden
% \usepackage{catoptions}

\makeatletter%
	\def\Autoref#1{%
		\begingroup%
			\def\reserved@a{\cpttrimspaces{#1}}%
			\ifcsndefTF{r@#1}{%
				\xaftercsname{\expandafter\testreftype\@fourthoffive}%
				{r@\reserved@a}.\\{#1}%
			}{%
				\ref{#1}%
			}%
		\endgroup%
	}%
	\def\testreftype#1.#2\\#3{%
		\ifcsndefTF{#1autorefname}{%
			\def\reserved@a##1##2\@nil{%
				\uppercase{\def\ref@name{##1}}%
				\csn@edef{#1autorefname}{\ref@name##2}%
				\autoref{#3}%
			}%
			\reserved@a#1\@nil%
		}{%
			\autoref{#3}%
		}%
	}%
\makeatother%

\usepackage{imakeidx}
\makeindex

\numberwithin{figure}{section}
\numberwithin{table}{section}
\numberwithin{table}{section}
\numberwithin{equation}{section}
\addto\extrasngerman{%
	\renewcommand{\figureautorefname}{Abbildung}
	\renewcommand{\tableautorefname}{Tabelle}
	\renewcommand{\subsectionautorefname}{Unterabschnitt}%
	\renewcommand{\sectionautorefname}{Abschnitt}%
	\renewcommand{\chapterautorefname}{Kapitel}%
	\renewcommand{\partautorefname}{Teil}%
}
\pagecolor{white}
\usepackage{pdfpages}
\newcommand{\imgbox}[7]{%
	\begin{wrapfigure}{#1}{#2}%
		\centering%
		\vspace{-0.5em}%
		\includegraphics[width=#3]{figures/#4}%
		\vspace{-1.0em}%
		\setcounter{figure}{\thethmhelper}%
		\addtocounter{thmhelper}{1}%
		\caption[#6]{#7}\label{fig:#5}%
		\vspace{-0.5em}%
	\end{wrapfigure}%
}%
\newcommand{\img}[5]{%
	\begin{figure}[htbp]%
		%% Eine normale Abbildung
		%%  #1 Breite der Abbildung
		%%  #2 Abbildung
		%%  #3 Label
		%%  #4 TOC-Beschriftung
		%%  #5 Beschriftung
		%%
		%% (Falls es komisch aussieht, setzen Sie
		%%  nur „t“ oder „tb“ statt „htbp“)
		\vspace{2.5em}%
		\centering%
		\includegraphics[width=#1]{figures/#2}
		\vspace{-0.5em}%
		\setcounter{figure}{\thethmhelper}%
		\addtocounter{thmhelper}{1}%
		\caption[#4]{#5}\label{fig:#3}%
		\vspace{1.5em}%
	\end{figure}%
}%
\newcommand\imgAB[9]{%
	%% LaTeX-Hexerei mit \imgABcontinued
	%% (wegen mehr als 9 Parametern)
	\def\tmpvalone{#1}%
	\def\tmpvaltwo{#2}%
	\def\tmpvalthr{#3}%
	\def\tmpvalfou{#4}%
	\def\tmpvalfiv{#5}%
	\def\tmpvalsix{#6}%
	\def\tmpvalsev{#7}%
	\def\tmpvaleig{#8}%
	\def\tmpvalnin{#9}%
	\imgABcontinued%
}%
\newcommand{\imgABcontinued}[2]{%
	\begin{figure}[htbp]%
	%% zwei Abbildungen nebeneinander
	%%  #1 Label des Containers
	%%  #2 TOC-Beschriftung des Containers
	%%  #3 Beschriftung des Containers
	%%  #4 Breite des linken Containers (A)
	%%  #5 Breite der Abbildung in A
	%%  #6 Abbildung in A
	%%  #7 Beschriftung in A
	%%  #8 Breite des rechten Containers (B)
	%%  #9 Breite der Abbildung in B
	%% #10 Abbildung in B
	%% #11 Beschriftung in B
	%%
	%% (Falls es komisch aussieht, setzen Sie
	%%  nur „t“ oder „tb“ statt „htbp“)
		\vspace{2.5em}%
		\centering%
		\begin{subfigure}[t]{\tmpvalfou\textwidth}%
			\centering%
			\includegraphics[width=\tmpvalfiv]{figures/\tmpvalsix}%
			\caption{\tmpvalsev}\label{fig:\tmpvalone-A}%
		\end{subfigure}%
		\hfill~%
		\begin{subfigure}[t]{\tmpvaleig\textwidth}%
			\centering%
			\includegraphics[width=\tmpvalnin]{figures/#1}%
			\caption{#2}\label{fig:\tmpvalone-B}%
		\end{subfigure}%
		\vspace{-0.5em}%
		\addtocounter{thmhelper}{1}%
		\setcounter{figure}{\thethmhelper}%
		\caption[\tmpvaltwo]{\tmpvalthr}\label{fig:\tmpvalone}%
		\vspace{1.5em}%
	\end{figure}%
}%
\newcommand{\imgABstack}[9]{%
	\begin{figure}[htbp]%
	%% zwei Abbildungen untereinander
	%% #1 Label des Containers
	%% #2 TOC-Beschriftung des Containers
	%% #3 Beschriftung des Containers
	%% #4 Breite der oberen Abbildung (A)
	%% #5 obere Abbildung (A)
	%% #6 Beschriftung der oberen Abbildung (A)
	%% #7 Breite der unteren Abbildung (B)
	%% #8 untere Abbildung (B)
	%% #9 Beschriftung der unteren Abbildung (B)
	%%
	%% (Falls es komisch aussieht, setzen Sie
	%%  nur „t“ oder „tb“ statt „htbp“)
		\vspace{2.5em}%
		\centering%
		\begin{subfigure}[t]{\textwidth}%
			\centering%
			\includegraphics[width=#4]{figures/#5}%
			\caption{#6}\label{fig:#1-A}%
		\end{subfigure}%
		\\%
		\vspace{2em}%
		\begin{subfigure}[t]{\textwidth}%
			\centering%
			\includegraphics[width=#7]{figures/#8}%
			\caption{#9}\label{fig:#1-B}%
		\end{subfigure}%
		\vspace{-0.5em}%
		\addtocounter{thmhelper}{1}%
		\setcounter{figure}{\thethmhelper}%
		\caption[#2]{#3}\label{fig:#1}%
		\vspace{1.5em}%
	\end{figure}%
}%
\newcommand\imgABC[9]{%
	%% LaTeX-Hexerei mit \imgABCcontinued
	%% (wegen mehr als 9 Parametern)
	\def\tmpvalone{#1}%
	\def\tmpvaltwo{#2}%
	\def\tmpvalthr{#3}%
	\def\tmpvalfou{#4}%
	\def\tmpvalfiv{#5}%
	\def\tmpvalsix{#6}%
	\def\tmpvalsev{#7}%
	\def\tmpvaleig{#8}%
	\def\tmpvalnin{#9}%
	\imgABCcontinued%
}%
\newcommand{\imgABCcontinued}[3]{%
	\begin{figure}[htbp]%
	%% drei Abbildungen nebeneinander
	%% #1 Label des Containers
	%% #2 TOC-Beschriftung des Containers
	%% #3 Beschriftung des Containers
	%% #4 Breite der linken Abbildung (A)
	%% #5 linke Abbildung (A)
	%% #6 Beschriftung der linken Abbildung (A)
	%% #7 Breite der mittleren Abbildung (B)
	%% #8 mittlere Abbildung (B)
	%% #9 Beschriftung der mittleren Abbildung (B)
	%% #10 Breite der rechten Abbildung (C)
	%% #11 rechte Abbildung (C)
	%% #12 Beschriftung der rechten Abbildung (C)
	%%
	%% (Falls es komisch aussieht, setzen Sie
	%%  nur „t“ oder „tb“ statt „htbp“)
		\vspace{2.5em}%
		\centering%
		\begin{subfigure}[t]{0.3\textwidth}%
			\centering%
			\includegraphics[width=\tmpvalfou]{figures/\tmpvalfiv}%
			\caption{\tmpvalsix}\label{fig:\tmpvalone-A}%
		\end{subfigure}%
		\hfill~%
		\begin{subfigure}[t]{0.3\textwidth}%
			\centering%
			\includegraphics[width=\tmpvalsev]{figures/\tmpvaleig}%
			\caption{\tmpvalnin}\label{fig:\tmpvalone-B}%
		\end{subfigure}%
		\hfill~%
		\begin{subfigure}[t]{0.3\textwidth}%
			\centering%
			\includegraphics[width=#1]{figures/#2}%
			\caption{#3}\label{fig:\tmpvalone-C}%
		\end{subfigure}%
		\vspace{-0.5em}%
		\addtocounter{thmhelper}{1}%
		\setcounter{figure}{\thethmhelper}%
		\caption[\tmpvaltwo]{\tmpvalthr}\label{fig:\tmpvalone}%
		\vspace{1.5em}%
	\end{figure}%
}%
\newcommand\imgABCstack[9]{%
	%% LaTeX-Hexerei mit \imgABCcontinued
	%% (wegen mehr als 9 Parametern)
	\def\tmpvalone{#1}%
	\def\tmpvaltwo{#2}%
	\def\tmpvalthr{#3}%
	\def\tmpvalfou{#4}%
	\def\tmpvalfiv{#5}%
	\def\tmpvalsix{#6}%
	\def\tmpvalsev{#7}%
	\def\tmpvaleig{#8}%
	\def\tmpvalnin{#9}%
	\imgABCcontinued%
}%
\newcommand{\imgABCstackcontinued}[3]{%
	\begin{figure}[htbp]%
	%% drei Abbildungen untereinander
	%% #1 Label des Containers
	%% #2 TOC-Beschriftung des Containers
	%% #3 Beschriftung des Containers
	%% #4 Breite der oberen Abbildung (A)
	%% #5 obere Abbildung (A)
	%% #6 Beschriftung der oberen Abbildung (A)
	%% #7 Breite der mittleren Abbildung (B)
	%% #8 mittlere Abbildung (B)
	%% #9 Beschriftung der mittleren Abbildung (B)
	%% #10 Breite der unteren Abbildung (C)
	%% #11 untere Abbildung (C)
	%% #12 Beschriftung der unteren Abbildung (C)
	%%
	%% (Falls es komisch aussieht, setzen Sie
	%%  nur „t“ oder „tb“ statt „htbp“)
		\vspace{2.5em}%
		\centering%
		\begin{subfigure}[t]{\textwidth}%
			\centering%
			\includegraphics[width=\tmpvalfou]{figures/\tmpvalfiv}%
			\caption{\tmpvalsix}\label{fig:\tmpvalone-A}%
		\end{subfigure}%
		\\%
		\vspace{2em}%
		\begin{subfigure}[t]{\textwidth}%
			\centering%
			\includegraphics[width=\tmpvalsev]{figures/\tmpvaleig}%
			\caption{\tmpvalnin}\label{fig:\tmpvalone-B}%
		\end{subfigure}%
		\\%
		\vspace{2em}%
		\begin{subfigure}[t]{\textwidth}%
			\centering%
			\includegraphics[width=#1]{figures/#2}%
			\caption{#3}\label{fig:\tmpvalone-C}%
		\end{subfigure}%
		\vspace{-0.5em}%
		\addtocounter{thmhelper}{1}%
		\setcounter{figure}{\thethmhelper}%
		\caption[\tmpvaltwo]{\tmpvalthr}\label{fig:\tmpvalone}%
		\vspace{1.5em}%
	\end{figure}%
}%
\newcommand\imgABCD[9]{%
	%% LaTeX-Hexerei mit \imgABCcontinued
	%% (wegen mehr als 9 Parametern)
	\def\tmpvalone{#1}%
	\def\tmpvaltwo{#2}%
	\def\tmpvalthr{#3}%
	\def\tmpvalfou{#4}%
	\def\tmpvalfiv{#5}%
	\def\tmpvalsix{#6}%
	\def\tmpvalsev{#7}%
	\def\tmpvaleig{#8}%
	\def\tmpvalnin{#9}%
	\imgABCcontinued%
}%
\newcommand{\imgABCDcontinued}[6]{%
	\begin{figure}[htbp]%
	%% vier Abbildungen in 2x2-Anordnung (Spalten je 50%)
	%% #1 Label des Containers
	%% #2 TOC-Beschriftung des Containers
	%% #3 Beschriftung des Containers
	%% #4 Breite der oberen, linken Abbildung (A)
	%% #5 obere, linke Abbildung (A)
	%% #6 Beschriftung der oberen, linken Abbildung (A)
	%% #7 Breite der oberen, rechten Abbildung (B)
	%% #8 obere, rechte Abbildung (B)
	%% #9 Beschriftung der oberen, rechten Abbildung (B)
	%% #10 Breite der unteren, linken Abbildung (C)
	%% #11 untere, linke Abbildung (C)
	%% #12 Beschriftung der unteren, linken Abbildung (C)
	%% #13 Breite der unteren, rechten Abbildung (D)
	%% #14 untere, rechten Abbildung (D)
	%% #15 Beschriftung der unteren, rechten Abbildung (D)
	%%
	%% (Falls es komisch aussieht, setzen Sie
	%%  nur „t“ oder „tb“ statt „htbp“)
		\vspace{2.5em}%
		\centering%
		\begin{subfigure}[t]{.45\textwidth}%
			\centering%
			\includegraphics[width=\tmpvalfou]{figures/\tmpvalfiv}%
			\caption{\tmpvalsix}\label{fig:\tmpvalone-A}%
		\end{subfigure}%
		\hfill~%
		\begin{subfigure}[t]{.45\textwidth}%
			\centering%
			\includegraphics[width=\tmpvalsev]{figures/\tmpvaleig}%
			\caption{\tmpvalnin}\label{fig:\tmpvalone-B}%
		\end{subfigure}%
		\\%
		\vspace{2em}%
		\begin{subfigure}[t]{.45\textwidth}%
			\centering%
			\includegraphics[width=#1]{figures/#2}%
			\caption{#3}\label{fig:\tmpvalone-C}%
		\end{subfigure}%
		\hfill~%
		\begin{subfigure}[t]{.45\textwidth}%
			\centering%
			\includegraphics[width=#4]{figures/#5}%
			\caption{#6}\label{fig:\tmpvalone-D}%
		\end{subfigure}%
		\vspace{-0.5em}%
		\addtocounter{thmhelper}{1}%
		\setcounter{figure}{\thethmhelper}%
		\caption[\tmpvaltwo]{\tmpvalthr}\label{fig:\tmpvalone}%
		\vspace{1.5em}%
	\end{figure}%
}%

\usepackage[tight]{minitoc}
\renewcommand{\ptcfont}{\small\normalfont}
\renewcommand{\mtcfont}{\small\normalfont}
\renewcommand{\mtcgapbeforeheads}{0pt}
\renewcommand{\mtcgapafterheads}{-40pt}
\mtcsetfeature{parttoc}{before}{\empty}
\mtcsetfeature{parttoc}{after}{\empty}
\renewcommand{\ptctitle}{Inhalte dieses Teils}
\renewcommand{\plftitle}{Abbildungen dieses Teils}
\renewcommand{\plttitle}{Tabellen dieses Teils}
\renewcommand{\mtctitle}{Inhalte dieses Kapitels}
\renewcommand{\mlftitle}{Abbildungen dieses Kapitels}
\renewcommand{\mlttitle}{Tabellen dieses Kapitels}
\mtcsetfeature{partlof}{before}{\empty}
\mtcsetfeature{partlof}{after}{\empty}
\mtcsetfeature{partlot}{before}{\empty}
\mtcsetfeature{partlot}{after}{\empty}

\usepackage{silence}
\WarningFilter{minitoc(hints)}{W0023}
\WarningFilter{minitoc(hints)}{W0024}
\WarningFilter{minitoc(hints)}{W0028}
\WarningFilter{minitoc(hints)}{W0030}
\WarningFilter{minitoc(hints)}{W0044}

\usepackage{fancyhdr}
\setlength{\headheight}{12.2pt}
\fancypagestyle{myfancy}{%
	\fancyhf{}
	\fancyhead[LE,RO]{\leftmark}
	% Die folgende Zeile einkommentieren, wenn Ihr Name auf jeder
	% Seite innen gesetzt werden soll.
	%\fancyfoot[RE,LO]{\myName}
	\fancyfoot[LE,RO]{\thepage}
	\renewcommand{\headrulewidth}{0.2pt}%
	\renewcommand{\footrulewidth}{0pt}%
}
\fancypagestyle{plain}{%
	\fancyhf{}%
	\fancyfoot[LE,RO]{\thepage}%
	\renewcommand{\headrulewidth}{0pt}%
	\renewcommand{\footrulewidth}{0pt}%
}

% Konfiguration der Absatzumbrüche
\setlength{\parindent}{0pt} % definiert Zeilenvorschub
\setlength{\parskip}{4pt}   % definiert Absatzdistanz
% Die Anweisung \noindent unterdrückt den Zeilenvorschub
% für den Fall, dass Sie einen Wert ungleich 0 für das
% \parindent definiert haben. Es ist üblich, dass der
% erste Absatz nach der Überschrift des Kapitels bzw.
% der Sektion nicht eingerückt wird.

\PassOptionsToPackage{hyphens}{url}
\usepackage{hyperref}
\usepackage{xr-hyper}
\usepackage{hyperxmp}
\hypersetup{%
	pdfmetalang		= {de-DE},
	pdfauthor		= {\myName},
	pdftitle		= {{\myTitle}},
	pdfsubject		= {Abschlussarbeit an der Berufsakademie Sachsen, \myAcademy},
	pdfcopyright	= {Copyright ©\mySubmissionDate\ by \myName.\012All rights reserved.},
	pdfcenterwindow	= {true},
	pdfpagelayout	= {TwoColumnRight},
	colorlinks		= {true},
	linkcolor		= {HKS41-100},
	citecolor		= {HKS57-100},
	filecolor		= {HKS07-100},
	urlcolor		= {HKS44-100},
}
\usepackage[all]{hypcap}

\usepackage[nameinlink,noabbrev]{cleveref}
\makeatletter
\newcommand\footnoteref[1]{\protected@xdef\@thefnmark{\ref{#1}}\@footnotemark}
\makeatother

%
% pagebackref erzeugt inverse Referenzen in der Literaturliste,
% d.h. es wird für jeden Eintrag eine Liste mit Seitenzahlen,
% auf denen eine Referenz gesetzt wurde, erzeugt.
%
% Damit das funktioniert MUSS jedes bibitem in der bib-Datei
% eine Leerzeile nach sich haben.
%
% Falls Sie BibLaTeX verwenden wollen, müssen Sie die folgenden
% Zeilen bis „und~}“ auskommentieren!
%
\renewcommand*{\backref}[1]{}
\renewcommand*{\backrefalt}[4]{%
	\ifcase #1 %
		\\(nicht explizit zitiert)%
	\or
		\\(zitiert auf Seite~#2)%
	\else
		\\(zitiert auf Seiten~#2)%
	\fi%
}
\renewcommand*{\backrefsep}{, }
\renewcommand*{\backreftwosep}{ und~}
\renewcommand*{\backreflastsep}{ und~}

\usepackage[shortcuts]{glossaries}
% shortcuts wird für \ac und \acp benötigt
\makeglossaries

%
% Zählvariablen für das Autorenreferat
%
\usepackage{refcount}
\usepackage{totcount}
\newtotcounter{citenum}
\def\oldcite{}
\let\oldcite=\bibcite
\def\bibcite{\stepcounter{citenum}\oldcite}
%
% Vorsicht! Wir benutzen hier eine LaTeX-Eigenschaft in Form von
% schwarzer Magie... Durch den Befehl \appendix in der thesis.tex
% wird der Kapitelzähler (chapter) zurückgesetzt. Dadurch enthält
% er am Ende nur die Anzahl Kapitel im Anhang.
% Wenn Sie keinen \appendix haben oder den Anhang durch \section
% anstatt \chapter organisieren, wird die Auswertung eine falsche
% Kapitelanzahl im Autorenreferat aufzeigen.
\regtotcounter{chapter}
%
% Ende Autorenreferat
%

%
% Wenn Sie lieber BibLaTeX verwenden wollen...
% Achtung: Backrefs funktionieren dann nicht mehr! Obiges
%          Auskommentieren nicht vergessen! Die Literaturliste
%          verweist dann nicht mehr auf die Seiten der
%          Verwendung...
%
% \usepackage[backend=biber,style=alphabetic,natbib=true,hyperref=true,backref=true]{biblatex}
% \DefineBibliographyStrings{ngerman}{
%    andothers = {{et\,al\adddot}},
%    backrefpage={zitiert auf Seite},
%    backrefpages={zitiert auf Seiten}
% }
% \bibliography{bibliography} % hier korrekte .bib-Datei verlinken
%
% Sie müssen auch die Option „pagebackref“ in der Dokumentklasse
% (in der thesis-tex) auskommentieren!