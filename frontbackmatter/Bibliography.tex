\renewcommand{\bibname}{Quellenverzeichnis}
\bibliographystyle{baalphadin}

\pdfbookmark[-1]{Quellenverzeichnis}{pdf-Bibliography}%
\phantomsection%
\addcontentsline{toc}{chapter}{Quellenverzeichnis}%
% BibTeX:
\bibliography{bibliography}\label{Bibliography}
% BibLateX:
% \printbibliography


\clearpage\section*{Hilfsmittel}\label{Tools}
Zur Erstellung der vorliegenden Abschlussarbeit wurden die folgenden Hilfsmittel verwendet:
% Seien Sie so spezifisch wie möglich!
\begin{itemize}
	\item ChatGPT --\,\url{https://chat.openai.com/chat},
	\item Comprehensive \TeX Archive Network, The --\,\url{http://ctan.org},
	\item dict.leo.org by LEO GmbH,
	\item Encyclop{\ae}dia Britannica,
	\item Gemeinsame Bibliothek der BA Dresden und der ehs Dresden,
	\item Google-Scholar --\,\url{https://scholar.google.com},
	\item Korrekturlesende (namentlich: Erika und Franz Mustermensch),
	\item Microsoft Office 365 University (2301 Build 16026.20200, x64),
	\item Microsoft Visio Professional 2019 (2301 Build 16026.20200, x64),
	\item Microsoft Windows 11 (22H2 Build 22621.1265, x64),
	\item Notepad++ 8.4.8 (32 Bit),
	\item Overleaf --\,\url{https://www.overleaf.com},
	\item Paint.NET (5.0.1, x86),
	\item S{\"a}chsische Landesbibliothek \mbox{--\,Staats}- und Universit{\"a}tsbibliothek Dresden, und
	\item Wiktionary.
\end{itemize}

\comment{
    Sie sollten bei Software eine möglichst abschließende und präzise Liste (inkl. Versionsnummer, etc.) angeben.}

\comment{
    Seien Sie bei Hardware so präzise wie möglich, wenn die Hardware Einfluss auf Messergebnisse o.\,ä. hat. Angaben zum Setup Ihrer Versuchsumgebung machen Sie bei der Beschreibung der Methodik (siehe \hyperref[comment:Setup]{Kommentar} in \autoref{chap:Evaluation}), nicht auf einer gesonderten Seite wie hier!
}

\comment{
    Sollte die verwendete Software Einfluss auf Ihre Messergebnisse haben, z.\,B. wenn Sie Usability-Tests unter Windows und MacOS durchführen, geben Sie die entsprechenden Informationen ebenfalls bei der Beschreibung der Methodik an (siehe \hyperref[comment:Setup]{Kommentar} in \autoref{chap:Evaluation}). --\,Die Liste auf dieser Seite beschränkt sich auf die Erstellung der Abschlussarbeit als Schriftwerk, nicht auf ihre wissenschaftlichen Inhalte.
}

\comment{
    Die URLs sind hier ohne Datum des letzten erfolgreichen Zugriffs aufgeführt, da bei den Hilfsmitteln davon ausgegangen wird, dass Sie diese bis zum Tag der Finalisierung Ihrer Ausarbeitung verwendet haben. Somit wird implizit das Unterschriftsdatum Ihrer \hyperref[Declaration]{Erklärung an Eides statt} als Zugriffsdatum angenommen.
}