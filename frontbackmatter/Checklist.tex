\pagestyle{empty}
\cleardoublepage
\pagenumbering{gobble}
\renewcommand{\thefootnote}{\alph{footnote}}
\setcounter{footnote}{0}
%%
%% Dieses Blatt ist nicht Bestandteil der Abschlussarbeit!
%% Entfernen Sie den Include der „Checklist.tex“ aus der
%% „thesis.tex“ bevor Sie Drucken gehen!
%%
\chapter*{Checkliste Abschlussarbeit}
In der folgenden Liste sind die wichtigsten, häufig nachgefragten Punkte rund um Ihre Abschlussarbeit zum Abhaken zusammengefasst.

\begin{itemize}
    \item[$\square$]{
        formale Kriterien mit beiden Begutachtenden klären
        \begin{itemize}
            \item[$\square$] Seitennutzung klären (einseitig oder doppelseitig)
            \item[$\square$] Zitationsstil und Referenzierstil klären
            \item[$\square$]{
                Sonstige Vorgaben klären (z.\,B. minimale Schriftgröße in Abbildungen)
            }
        \end{itemize}
    }
    \item[$\square$]{
        Anzahl Print-Exemplare klären
        \begin{itemize}
            \item[$\square$] Pflicht-Exemplar für Archiv (\emph{muss} ein Print-Exemplar sein)
            \item[$\square$] begutachtende Person beim Praxispartner
            \item[$\square$] begutachtende Person an der Berufsakademie
        \end{itemize}
    }
    \item[$\square$]{
        Form der elektronischen Abgabe klären\\
        (die Prüfungsordnung macht hier keine Vorgabe!)
        \begin{itemize}
            \item[$\square$] Pflicht-Exemplar für Archiv (z.\,B. selbst gebrannte DVD)
            \item[$\square$] begutachtende Person beim Praxispartner
            \item[$\square$] begutachtende Person an der Berufsakademie
        \end{itemize}
    }
    \item[$\square$]{
        Auftragsblatt
        \begin{itemize}
            \item[$\square$] Auftragsblatt einscannen
            \item[$\square$] Original des Auftragsblatt in ein Print-Exemplar einbinden
            \item[$\square$] Kopie des Auftragsblatts in die anderen Print-Exemplare einbinden
            \item[$\square$] Scan des Auftragsblatts in elektronischer Version einfügen
        \end{itemize}
    }
    \item[$\square$] Alle referenzierten Quellen auf Quelleneigenschaften prüfen\footnote{Alles, was nicht den Quelleneigenschaften genügt, sollten Sie in Fußnoten auslagern.}
    \item[$\square$] Notwendigkeit des Thesenblatts klären\footnote{Weder in der Prüfungsordnung noch dem Modulhandbuch existiert das Thesenblatt.}
    \item[$\square$]{
        Zusammenfassung der Arbeit mit allen wichtigen Ergebnissen schreiben
        \begin{itemize}
            \item[$\square$] geschriebene Zusammenfassung in Abstract übernehmen
            \item[$\square$] geschriebene Zusammenfassung leicht abwandeln und in Thesenblatt übernehmen
        \end{itemize}
    }
    \item[$\square$]{
        Forschungsfragen und Thesen
        \begin{itemize}
            \item[$\square$] Prüfen, ob Forschungsfragen tatsächlich in der Ausarbeitung beantwortet werden
            \item[$\square$] Falls keine Forschungsfragen: Prüfen, ob Hypothesen in der Ausarbeitung bestätigt oder widerlegt werden
            \item[$\square$] Bei widerlegten Thesen: Herleitung von Antithesen und/oder Synthesen prüfen
            \item[$\square$] Falls Thesenblatt notwendig (s.o.):  Forschungsfragen und/oder Tehsen auf das Thesenblatt übertragen
        \end{itemize}
    }
    \clearpage
    \item[$\square$]{
        Druck und Abgabe vorbereiten
        \begin{itemize}
             \item[$\square$]{
                Beim Druckgeschäft des Vertrauens nach Druckkapazitäten fragen\\
                (Neben Ihnen wollen viele andere Studierende ihre Abschlussarbeiten drucken!)
                \begin{itemize}
                    \item[$\square$] Anfragen Kapazitäten in der Woche der Abgabe
                    \item[$\square$] Spätesten Abgabetermin PDF im Geschäft klären, damit Wunsch-Fer\-tig\-stel\-lungs\-ter\-min gehalten werden kann
                    \item[$\square$] Kosten Graustufen- vs. Farbdruck klären
                    \item[$\square$]{
                        archivierbare Bindung klären\\
                        (Das Pflichtexemplar für das Archiv der Berufsakademie muss langzeit-ar\-chi\-vier\-bar sein und darf keine Klebstoffe in der Bindung enthalten)
                    }
                \end{itemize}
            }
            \item[$\square$]{
                Im Service-Büro nachfragen:
                \begin{itemize}
                    \item[$\square$] Bürozeiten und/oder Vertretungsregelung, falls Sie persönlich abgeben wollen
                    \item[$\square$] korrekte Adressierung für postalische\footnote{Beachten Sie, dass die Deutsche Post AG lediglich Zustellungen binnen $3$ Werktagen für $95\,\%$ der Sendungen garantiert.} Einreichung\footnote{Einschreiben und Rückschein bringen juristisch nichts, da diese nur nachweisen, dass Sie \emph{etwas} eingesendet haben, aber nicht \emph{was}. Wenn Sie rechtssicher auch den Inhalt der Einsendung, nämlich die Print-Exemplare, nachweisen wollen, müssen Sie einen Gerichtsvollzieher beauftragen! Achtung: Teuer.}
                    \item[$\square$] Schriftliche Bestätigung der fristwahrenden Einreichung
                \end{itemize}
            }
            \item[$\square$]{\color{red}
                Diese Checkliste aus der Ausarbeitung entfernen (siehe Kommentar zu Beginn des Quellkodes; auf keinen Fall mit einbinden und/oder abgeben!)
            }
            \item[$\square$]{
                Unterschriften und Daten prüfen
                \begin{itemize}
                    \item[$\square$]{
                        korrektes Abgabedatum auf/in der Abschlussarbeit
                        \begin{itemize}
                            \item[$\square$] $\leq$ späteste Abgabe laut Auftragsblatt (bzw. genehmigter Verlängerung)
                            \item[$\square$] Bei genehmigter Verlängerung: ursprüngliches spätestes Abgabedatum laut Auftragsblatt in Klammern hinter Abgabedatum
                        \end{itemize}
                    }
                    \item[$\square$] Datum der Erklärung an Eides statt $\leq$ Abgabedatum
                    \item[$\square$]{
                        Erklärung an Eides statt unterschreiben
                        \begin{itemize}
                            \item[$\square$] Print-Exemplar(e)
                            \item[$\square$] PDF-Datei
                        \end{itemize}
                    }
                \end{itemize}
            }
        \end{itemize}
    }
    \item[$\square$]{
        Verteidigung vorbereiten
        \begin{itemize}
            \item[$\square$]{
                Begutachtende nach Kritikpunkten und Fragen aus den Gutachten fragen\footnote{Die Gutachten müssen laut Prüfungsordnung binnen $4$ Wochen nach Abgabe vorliegen; laut Rahmengesetz haben Sie sogar einen Anspruch auf Einsicht in die Gutachten (Einsicht, nicht Kopie/Scan!), spätestens $2$ Wochen vor dem Verteidigungstermin; im Moment ignoriert die Berufsakademie diesen Rechtsanspruch, also formulieren Sie das Ganze als freundliche Anfrage an die Begutachtenden}
                \begin{itemize}
                    \item[$\square$] Gutachten Praxispartner
                    \item[$\square$] Gutachten Berufsakademie
                \end{itemize}
            }
            \item[$\square$]{
                Auf Basis der Kritikpunkte und Fragen ggf. inhaltlich nachbessern.\\
                (Sie sollten in der Verteidigung gezielt auf Kritik eingehen und Verbesserungen präsentieren, z.\,B. \enquote{Im Gutachten wurde die Nutzerstudie in der Evaluation bemängelt. Ich habe inzwischen eine systemisch verbesserte Befragung durchgeführt; die Ergebnisse konnten im Rahmen des Fehlermaßes reproduziert und somit bestätigt werden.})
            }         
            \item[$\square$] Notwendigkeit des Posters für die Verteidigung klären\footnote{Weder in der Prüfungsordnung noch im Modulhandbuch existiert das Poster.}
            \item[$\square$]{
                Möglichkeit der Prüfung der Präsentation vor der Verteidigung klären
                \begin{itemize}
                    \item[$\square$] Kollegium, Freundeskreis, Familie, $\ldots$
                    \item[$\square$] begutachtende Person beim Praxispartner
                    \item[$\square$] begutachtende Person an der Berufsakademie
                \end{itemize}
            }
        \end{itemize}
    }
\end{itemize}