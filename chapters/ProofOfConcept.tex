\chapter[Implementierung / Proof of Concept]{Implementierung /\newline Proof of Concept}\label{chap:ProofOfConcept}
Nachweis der Umsetzbarkeit Ihres Konzeptes, in der Regel durch eine prototypische Implementierung oder einen Nachweis durch Probandenbefragungen et cetera.

Beschreiben Sie die \enquote{Highlights} Ihres Proof-of-Concepts. Der gesamte Quellkode, die Konstruktionspläne Ihres Modells, die vollständigen CAD-basierten Berechnungen etc. sind als Ganzes uninteressant. Konzentrieren Sie sich auf besonders interessante Probleme und wie Sie sie gelöst haben. Abenteuerliches Beispiel: Sie haben in Ihrem Konzept einen Mechanismus, der das Halteproblem löst. Im Proof-of-Concept haben Sie festgestellt, dass ein NP-harter Isomorphismus ausreicht. Zeigen Sie also die paar Kodezeilen, die diesen Isomporphismus umsetzen. ;-)

\comment{Das Kapitel heißt entweder \enquote{Implementierung} \emph{oder} \enquote{Proof of Concept}.\\--\,Nicht wie o.\,g. beides!}

\comment{An der Kapitelüberschrift sehen Sie übrigens wie Sie im Inhaltsverzeichnis einen anderen Text (hier \underline{ohne} Zeilenumbruch) wiedergeben als auf der eigentlichen Kapitelseite (hier \underline{mit} Zeilenumbruch).}

\comment{Bei der Implementierung von Software sollten Sie sich an \cite{sommerville1992} orientieren!}