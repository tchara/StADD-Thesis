\chapter{Eigenes Konzept}\label{chap:Concept}
Beschreiben Sie auf Basis der Ergebnisse Ihrer Literaturrecherche Ihr eigenes Konzept. Dazu ist in einem vorherigen Schritt (am Ende von \autoref{chap:RelatedWork}) oder direkt zu Beginn dieses Konzept-Kapitels eine Anforderungsanalyse sinnvoll. Am Ende der Anforderungsanalyse sollte eine Tabelle stehen, in welcher die Anforderungen nach Wichtigkeit gruppiert aufgeführt sind. Die gleiche Tabelle mit zwei zusätzlichen Spalten sollte sich im \autoref{chap:Results} wieder finden. Die beiden zusätzlichen Spalten sollten mit Häckchen ($\surd$) oder Kreuz ($\times$) die Erfüllung der jeweiligen Anforderung aufzeigen und, im $\surd$-Fall die Güte (\enquote{Falls \enquote{Ja}, wie gut?}), im $\times$-Fall die Gründe (\enquote{Falls \enquote{Nein}, weshalb nicht?}), aufführen. Um die Frage \enquote{Falls \enquote{Ja}, wie gut?} beantworten zu können, sollten Sie mit der Anforderungsanalyse auch eine Evaluationsmethodik und die zugehörige Evaluationsmetrik definieren.