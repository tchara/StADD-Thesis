\chapter{Grundlagen und verwandte Arbeiten}\label{chap:RelatedWork}
Beschreiben Sie in diesem Kapitel die Ergebnisse Ihrer Literaturrecherche.

\section{Grundlagen}\label{sec:RelatedWork:Foundations}
Falls Sie fundamentale Grundlagen beschreiben müssen, fügen Sie diese entsprechend hier vor der Literaturrecherche ein.

\subsection{Mathematische Grundlagen}\label{subsec:RelatedWork:Foundations:Maths}
Sollten Sie auf wichtige mathematische Grundlagen eingehen (beispielsweise weil Ihr toller neuer Kryptografiealgorithmus eine Erweiterung des Chinesischen Restsatzes voraussetzt), fügen Sie diese Grundlagen ebenfalls vor der Literaturrecherche ein.

\subsection{Technische Grundlagen}\label{subsec:RelatedWork:Foundations:Techs}
Technische Grundlagen wie sie beispielsweise in DIN, ISO, etc. beschrieben sind, sollten Sie auch getrennt aufführen.

\section{Verwandte Arbeiten}\label{sec:RelatedWork:Publications}
Der Teil \enquote{Verwandte Arbeiten} (in englischer Literatur i.d.R. \emph{Related Work}) betrachtet alle relevanten wissenschaftlichen Literaturquellen.

\section{Stand der Technik}\label{sec:RelatedWork:SotA}
Von wissenschaftlichen Quellen sollten Sie den Stand der Technik (in englischer Literatur i.d.R. \emph{State of the Art} oder \emph{SotA}) streng trennen. Hier betrachten Sie existierende Lösungen am Markt, etc.

\subsection{Best Practices}\label{subsec:RelatedWork:BestPractices}
Sollten sich aus der Marktanalyse klare Best Practices ergeben, fassen Sie diese am besten noch einmal separat zusammen.