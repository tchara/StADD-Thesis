\chapter{Grundlagen und verwandte Arbeiten}\label{chap:RelatedWork}
Beschreiben Sie in diesem Kapitel die Ergebnisse Ihrer Literaturrecherche.

Damit Sie strukturiert recherchieren und nicht den Überblick verlieren bieten sich Literaturveraltungsprogramme an. Oft fallen in diesem Zusammenhang die Namen Citavi\footnote{\url{https://www.citavi.com/} --\,zuletzt am 9. Oktober 2023 erfolgreich abgerufen} und Zotero\footnote{\url{https://www.zotero.org} --\,zuletzt am 9. Oktober 2023 erfolgreich abgerufen}. Einen sehr guten Online-Kurs zur Erfassung, Verwaltung und Zitation von Literatur bietet die SLUB\footnote{Sächsische Landesbibliothek -- Staats- und Universitätsbibliothek Dresden} in einem OPAL-Kurs\footnote{\url{https://bildungsportal.sachsen.de/opal/auth/RepositoryEntry/23157637126/CourseNode/98267014615921?10} --\,zuletzt am 9. Oktober 2023 erfolgreich abgerufen}.

\section{Grundlagen}\label{sec:RelatedWork:Foundations}
Falls Sie fundamentale Grundlagen beschreiben müssen, fügen Sie diese entsprechend hier vor der Literaturrecherche ein.

\subsection{Mathematische Grundlagen}\label{subsec:RelatedWork:Foundations:Maths}
Sollten Sie auf wichtige mathematische Grundlagen eingehen (beispielsweise weil Ihr toller neuer Kryptografiealgorithmus eine Erweiterung des Chinesischen Restsatzes voraussetzt), fügen Sie diese Grundlagen ebenfalls vor der Literaturrecherche ein.

\subsection{Technische Grundlagen}\label{subsec:RelatedWork:Foundations:Techs}
Technische Grundlagen wie sie beispielsweise in DIN, ISO, etc. beschrieben sind, sollten Sie auch getrennt aufführen.

\section{Verwandte Arbeiten}\label{sec:RelatedWork:Publications}
Der Teil \enquote{Verwandte Arbeiten} (in englischer Literatur i.d.R. \emph{Related Work}) betrachtet alle relevanten wissenschaftlichen Literaturquellen.

Meistens finden Sie passende Literatur bei Google Scholar\footnote{\url{https://scholar.google.com} --\,zuletzt am 30. September 2023 erfolgreich abgerufen} oder Research Gate\footnote{\url{https://researchgate.net} --\,zuletzt am 30. September 2023 erfolgreich abgerufen}. Oftmals werden Sie von dort zu Conference Proceedings oder Journalen verwiesen (unter Angabe eines DOI\footnote{Digital Object Identifier}). Leider sind mehr oder weniger aktuelle Paper oft hinter einer Paywal. Sie können versuchen, mittels Sci-Hub\footnote{\url{https://sci-hub.hkvisa.net} --\,zuletzt am 30. September 2023 erfolgreich abgerufen} die Paywal (legal) zu umgehen.

\section{Stand der Technik}\label{sec:RelatedWork:SotA}
Von wissenschaftlichen Quellen sollten Sie den Stand der Technik (in englischer Literatur i.\,d.\,R. \emph{State of the Art} oder \emph{SotA}) streng trennen. Hier betrachten Sie existierende Lösungen am Markt, etc.

\subsection{Best Practices}\label{subsec:RelatedWork:BestPractices}
Sollten sich aus der Marktanalyse klare Best Practices ergeben, fassen Sie diese am besten noch einmal separat zusammen.